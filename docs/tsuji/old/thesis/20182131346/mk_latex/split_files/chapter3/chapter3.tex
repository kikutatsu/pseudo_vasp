\chapter{はじめに}\label{ux306fux3058ux3081ux306b}

    \section{背景}\label{ux80ccux666f}

    西谷研では最安定の粒界エネルギーを第一原理計算で求める研究を行っている.

この研究において,
第一原理計算はVASPという計算ソフトによって自動で行われるが,その前後の作業工程のいくつかが手動で行われている.

\begin{quote}
主な作業名称:コマンド名(自動化の度合い)
\end{quote}
をまとめると 1. 原子モデル作成:modeler 1.
粒界セルモデル作成:make\_all(自動化済み) 1. 原子削除(手動) 1.
計算サーバへのファイル転送:scp(手動) 1. 計算設定ファイル:vasprun 1.
ファイル配置(自動化済み) 1. 構造最適化の手動設定(手動) 1.
第一原理計算:vasp(自動) 1. 結果の解析:rake gets finishedn(自動化済み)

である.

これらは使い慣れた作業者に取っては,
間違った場合もすぐに気づくことができ,
間違いのケアも迅速に出来るという点では良い.
しかし,初心者がこれらの作業を手動でやると,
途中で何をしているのか分からなくなり,効率が悪くなってしまう

\section{目的}\label{ux76eeux7684}

これらの手順の一部を自動で行ったり,
間違いを検出してくれるようなシステムを構築し,
初心者でも簡単に最安定な粒界エネルギーを求められるようにすることが本研究の目的である.

最初に構造最適化についておこなう.
さらに,自動原子削除についての試みを記す.
なお,作業全体の手順は藤村がまとめている{[}参照:藤村{]}.

    